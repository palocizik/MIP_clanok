% Metódy inžinierskej práce

\documentclass[10pt,twoside,slovak,a4paper]{article}

\usepackage[slovak]{babel}
%\usepackage[T1]{fontenc}
\usepackage[IL2]{fontenc} % lepšia sadzba písmena Ľ než v T1
\usepackage[utf8]{inputenc}
\usepackage{graphicx}
\usepackage{url} % príkaz \url na formátovanie URL
\usepackage{hyperref} % odkazy v texte budú aktívne (pri niektorých triedach dokumentov spôsobuje posun textu)

\usepackage{cite}
\usepackage{times}

\pagestyle{headings}

\title{Použitie hier a VR pri rehabilitácii\thanks{Semestrálny projekt v predmete Metódy inžinierskej práce, ak. rok 2022/23, vedenie: Ing. Igor Stupavský}} % meno a priezvisko vyučujúceho na cvičeniach

\author{Pavol Čížik\\[2pt]
	{\small Slovenská technická univerzita v Bratislave}\\
	{\small Fakulta informatiky a informačných technológií}\\
	{\small \texttt{xcizik@stuba.sk}}
	}

\date{\small 30. september 2022} 



\begin{document}

\maketitle

\begin{abstract}
Vo svojom článku by som sa chcel venovať téme použitie virtuálnej realita a gamifikácie pri rehabilitáciach. Takéto rehabilitácie by sa mohli použiť po zlomeninách alebo u pacientov po mŕtvici. Pretože medzi následky mŕtvice patrí nemožnosť normálnych kontrolovaných pohybov a problémy s hmatom a jemnou motorikou. 

V článku by som sa chcel zamerať na výhody a nevýhody, ktoré by takáto liečba priniesla. Pozrieť sa na ceny virtuálnych realít. A či by bolo možné rehabilitovať aj doma. 

Hry a virtuálna realita by mohli pacientov viacej motivovať do cvičenie čo by mohlo mať za dôsledok rýchlejšie zotavenie. Preto si myslím že táto téma má veľký význam a v budúcnosti by mohli hry a virtuálna realita výrazne pomôcť pri rehabilitáciach.
\end{abstract}



\section{Úvod}
...



\section{Použitie v praxi} \label{nejaka}
......



\section{Výhody} \label{ina}
.......



\section{Nevýhody} \label{dolezita}
.........\cite{7926560}



\section{Možné problémy} \label{dolezitejsia}
......



\section{Reakcia na témy z prednášok}
\paragraph{1}

.......
\paragraph{2}


.......
\paragraph{3}

.......



\section{Záver} \label{zaver} % prípadne iný variant názvu
..........


%\acknowledgement{Ak niekomu chcete poďakovať\ldots}


% týmto sa generuje zoznam literatúry z obsahu súboru literatura.bib podľa toho, na čo sa v článku odkazujete
\bibliography{literatura}
\bibliographystyle{plain} % prípadne alpha, abbrv alebo hociktorý iný
\end{document}
